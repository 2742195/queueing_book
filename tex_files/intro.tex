\chapter*{Introduction}\label{sec:introduction}
\addcontentsline{toc}{chapter}{Introduction}

\paragraph{Motivation and Examples}
Queueing systems abound, and the analysis and control of queueing
systems are major topics in the control, performance evaluation and
optimization of production and service systems. 


At my local supermarket, for instance, any customer that joins a queue
of 4 or more customers gets his/her groceries for free. Of course, there
are some constraints: at least one of the cashier facilities has to be
unoccupied by a server and the customers in queue should be equally
divided over the cashiers that are open. (And perhaps there are some
further rules, of which I am unaware.) The manager that controls the
occupation of the cashier positions is focused on keeping
$\pi(4)+\pi(5)+\cdots$, i.e., the fraction of customers that see upon
arrival a queue length exceeding~3, very small. In a sense,
this is easy enough: just hire many cashiers. However, the cost of
personnel may then outweigh the yearly average cost of paying the
customer penalties. Thus, the manager's problem becomes to plan and
control the service capacity  in such a way that both
the penalties and the personnel cost are small.

Fast food restaurants also deal with many interesting queueing
situations. Consider, for instance, the preparation of
hamburgers. Typically, hamburgers are made-to-stock, in other words,
they are prepared before the actual demand has arrived. Thus, hamburgers
in stock can be interpreted as customers in queue waiting for service,
where the service time is the time between the arrival of two
customers that buy hamburgers. The hamburgers have a typical lifetime,
and they have to be scrapped if they remain on the shelf longer than
some amount of time. Thus, the waiting time of hamburgers has to be
closely monitored. Of course, it is easy to achieve zero scrap cost,
simply by keeping no stock at all.  However, to prevent lost-sales it
is very important to maintain a certain amount of hamburgers in
stock. Thus, the manager has to balance the scrap cost against the
cost of lost sales. In more formal terms, the problem is to choose a
policy to prepare hamburgers such that the cost of excess waiting time
(scrap) is balanced against the cost of an empty queue (lost sales).

Service systems, such as hospitals, call centers, courts, and so on,
have a certain capacity available to serve customers. The performance
of such systems is, in part, measured by the total number of jobs
processed per year and the fraction of jobs processed within a certain
time between receiving and closing the job. Here the problem is to
organize the capacity such that the sojourn time, i.e., the typical
time a job spends in the system, does not exceed some threshold, and
such that the system achieves a certain throughput, i.e., jobs served
per year. 

Clearly, all the above systems can be seen as queueing systems that
have to be monitored and controlled to achieve a certain
performance. The performance analysis of such systems can, typically,
be characterized by the following performance measures:
\begin{enumerate}
\item The fraction of time $p(n)$ that the system contains $n$
  customers. In particular, $1-p(0)$, i.e., the fraction of time the
  system contains jobs, is important, as this is a measure of the
  time-average occupancy of the servers, hence related to personnel
  cost.
\item The fraction of customers $\pi(n)$ that `see upon arrival' the
  system with $n$ customers. This measure relates to customer
  perception and lost sales, i.e., fractions of arriving customers
  that do not enter the system.
\item The average, variance, and/or distribution of the waiting time.
\item The average, variance, and/or distribution of the number of customers in the system.\
\end{enumerate}
Here the system can be anything that is capable of holding jobs, such
as a queue, the server(s), an entire court house, patients waiting for
an MRI scan in a hospital, and so on.

It is important to realize that a queueing system can, typically, be
decomposed into \emph{two subsystems}, the queue itself and the
service system. Thus, we are concerned with three types of waiting:
waiting in queue, i.e., \emph{queueing time}, waiting while being in
service, i.e., the \emph{service time}, and the total waiting time in
the system, i.e., the \emph{sojourn time}.

\paragraph{Organization}


In these notes we will be primarily concerned with making models of
queueing systems such that we can compute or estimate the above
performance measures.  Part of our work is to derive analytic
models. The benefit of such models is that they offer structural
insights into the behavior of the system and scaling laws, such as
that the average waiting time scales (more or less) linearly in the
variance of the service times of individual customers. However, these
models have severe shortcomings when it comes to analyzing real
queueing systems, in particular when particular control rules have to
be assessed.  Consider, for example, the service process at a check-in
desk of KLM. Business customers and economy customers are served by
two separate queueing systems. The business customers are served by
one server, server A say, while the economy class customers by three
servers, say. What would happen to the sojourn time of the business
customers if server A would be allowed to serve economy class
customers when the business queue is empty? For the analysis of such complicated
cases simulation is a very useful and natural approach.

In the first part of these notes we concentrate on the analysis of
\emph{sample paths of a queueing process}. We assume that a typical
sample path captures the `normal' stochastic behavior of the
system. This sample-path approach has two advantages. In the first
place, most of the theoretical results follow from very concrete
aspects of these sample paths. Second, the analysis of sample-paths
carries over right away to simulation. In fact, simulation of a
queueing system offers us one (or more) sample path(s), and based on
such sample paths we derive behavioral and statistical properties of
the system. Thus, the performance measures defined for sample paths
are precisely those used for simulation.  

Our aim is not to provide rigorous proofs for all results derived below; for this we refer to
\cite{el-taha98:_sampl_path_analy_queuein_system}. As a consequence we tacitly
assume in the remainder that results derived from the (long-run)
analysis of a particular sample path are equal to their `probabilistic
counterpart'. 
% For instance, assume that the sequence of service times
% $S_1, S_2, \ldots$ for jobs $1, 2, \ldots$, are i.i.d., i.e.,
% independent and identically distributed. It then follows from the
% strong law of large numbers that $k^{-1}\sum_{i=1}^k S_i \to \E S$,
% almost surely, where $\E S$ is the expectation of a generic service
% time $S$ with the same distribution as the service times
% $\{S_i\}$. While this is already a quite deep result in probability
% theory, the problems become much harder when we have to consider the
% sequence of waiting times $\{W_i\}$. As we will see later, the times
% are constructed in terms of recursions, and are \emph{not}
% i.i.d. Thus, we cannot right away apply the strong law to conclude
% that results of sample path analysis capture the long-run
% probabilistic behavior the waiting time. However, intuitively, it is
% hopefully clear that the sequence of waiting times $W_i$ will
% converge, in some sense, to a random variable $W$ and that the
% stochastic properties of $W$ can be obtained from a sample-path
% analysis of the waiting times $\{W_i\}$. In the remainder we will
% freely use that $k^{-1}\sum_{i=1}^k W_i \to \E W$ as $k\to\infty$, and
% so on.

In the second part of these notes we construct algorithms to analyze open and closed
queueing networks. Many of the sample path results developed for the
single-station case can be applied to these networks. As such, theory,
simulation and algorithms form a nicely rounded out part of work.  For
this part we refer to the book of Prof. Zijm; the present set of notes
augments the discussion there.


\paragraph{Exercises}

I urge you to make \emph{all} exercises in this set of notes.
The main text contains hardly any examples or derivations: the exercises \emph{illustrate} the material and force you to \textit{think} about the technical parts.
The exercises require many of the tools you learned previously in courses on calculus, probability, and linear algebra.
Here you can see them applied.
Moreover, many of these tools will be useful for other, future, courses.
Thus, the investments made here will pay off for the rest of your (student) career.


As a guideline to making the exercises I recommend the following approach.
First read the notes.
Then attempt to make an exercise for 10 minutes or so by yourself.
If by that time you have not obtained a good idea on how to approach the problem, check the solution manual.
Once you have understood the solution, try to repeat the arguments \emph{with the solution manual closed}.

You'll notice that some these problems are quite difficult, often not because the problem itself is difficult, but because you need to combine a substantial amount of knowledge all at the same time.
All this takes time and effort.
Realize that the exercises are not intended to be easy (otherwise we could have been satisfied with computing $1+1$).
The problems should be doable, but hard.

The book is, admittedly, pretty big.
One reason is that the hints and solutions are very explicit, and spell out many intermediate steps.
For most of you all this detail is not necessary, but over the years I got many questions like: "how do you go from `here' to `there'?"
As service I then added such intermediate steps.
A second reason is that for some questions I included different ways to obtain the answer.
Thirdly, the numerical calculations show each intermediate numerical result and the code.
Like this, if you get stuck somewhere in the computations you can precisely check where you go wrong.
Finally, I believe that to understand a topic, it is necessary to study many examples, which in this case are covered in the form of exercises.

The solution manual is meant to prevent you from getting stuck and to help you increase your knowledge of probability, linear algebra, programming (analysis with computer support), and queueing in particular.
Thus, read the solutions very carefully.
Here and there I also included hints.




% \paragraph{Symbols}

% The meaning of the symbols in the margin of pages are as follows:
% \begin{itemize}
% \item The symbol \recall{} in the margin means that you have to
%   memorize the \emph{emphasized concepts}.
% \item The symbol \hintsymbol in the margin means that this question
%   has a \emph{hint}.
% \item The symbol \tbd in the margin means that this question or its
%   solution requires still some \emph{work on my part}; you can skip
%   it.
% \end{itemize}

\paragraph{Further reading}

The following books are very useful to extend one's knowledge of probability theory and queueing systems:
\begin{enumerate}
\item \citet{capinski03:_probab_probl}
\item \citet{tijms94:_stoch_model_algor_approac} and/or \citet{tijms03:_first_cours_stoch_model}
\item \citet{el-taha98:_sampl_path_analy_queuein_system}
\item \citet{bolch06:_queuein_networ_markov_chain}
\end{enumerate}

\paragraph{Acknowledgments}

I would like to acknowledge dr.
J.W.
Nieuwenhuis for our many discussions on the formal aspects of queueing theory, prof.
dr.
W.H.M.
Zijm for allowing me to use the first few chapters of his book, and my students for finding and correcting errors and typos, and improving explanations.

%\clearpage

%Finally, we use the terms `jobs' and `customers' interchangeably.

